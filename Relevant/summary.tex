\chapter{Executive Summary}

In this penetration test the Relevent medium level box
on tryhackme was examined
 for security-relevant weaknesses.
  The kind of testing was black-box,
   this is the kind where no specific
    information about the internals
	 of the system is given.
	  The scope of the assessment is as follows:
\begin{itemize}
	\item Dedicated Web Server: 10.10.212.187
\end{itemize}

Table \ref{tbl:web-sites} contains the overview of examined systems during the penetration test.
\begin{table}[h]
	\centering
	\begin{tabular}{|l|l|}
		\hline 
		\textbf{Services} & \textbf{Hostname}\\
		\hline 
		Website1\label{site1} & http://10.10.212.187/\\
		\hline 
		Website2\label{site2} & http://10.10.212.187:49663/\\
		\hline
		Smbserver\label{smb} & 10.10.212.187:445\\
		\hline
	\end{tabular}
	\caption{Web sites examined during the penetration test}
	\label{tbl:web-sites}
\end{table}

Several vulnerabilities have
been found among the assets of the organization,
some of them pose a significant risk.
%    Figure \ref{fig:vuln-by-type}
%     summarizes all issues by their type across
% 	 all the assets of Company X.
Solutions to remedy the discovered vulnerabilities
are provided together with detailed descriptions
and reproduction steps.
Detailed scan revealed an smbshare and webserver running IIS default
webpage. The smbshare allowed anonymous authentication with
reaad and write permissions and had
a password file in it. This particular shared folder was also
accessible through the website. This gives us the ability 
to execute code on server terminal. Checking the privileges
of the user, he has 'SeImpersonateToken' privilege enabled
which allows him to become Administrator on the machine and
take control of the whole machine. The smb server was also outdated and could
be exploited with the famous 'Eternal Blue' exploit to get
control of the machine. This is a serious vulnerability and needs
to be patched immediately.
% In this part add a short summary of all vulnerabilities in non-technical terms.

% It's also good to mention an estimation of efforts required to resolve the issues.